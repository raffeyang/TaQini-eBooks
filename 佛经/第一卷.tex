\section{般若波罗蜜多心经 · 简体}
观自在菩萨行深般若波罗蜜多时,照见五蕴皆空,度一切苦厄。

「舍利子!色不异空,空不异色,色即是空,空即是色;

受、想、行、识,亦复如是。

「舍利子!是诸法空相,不生不灭,不垢不净,不增不减。

是故,空中无色,无受、想、行、识;

无眼、耳、鼻、舌、身、意;无色、声、香、味、触、法;

无眼界,乃至无意识界;

无无明亦无无明尽,乃至无老死亦无老死尽;

无苦、集、灭、道;无智,亦无得。

「以无所得故,菩提萨,埵依般若波罗蜜多故,心无挂碍;

无挂碍故,无有恐怖,远离颠倒梦想,究竟涅槃。

三世诸佛依般若波罗蜜多故,得阿耨多罗三藐三菩提。

「故知般若波罗蜜多是大神咒、是大明咒、是无上咒、是无等等咒,

能除一切苦,真实不虚,故说般若波罗蜜多咒」。

即说咒曰:「揭帝 揭帝 般罗揭帝 般罗僧揭帝 菩提僧莎诃」。

\section{般若波羅蜜多心經 · 繁體}
觀自在菩薩行深般若波羅蜜多時,照見五蘊皆空,度一切苦厄。

「舍利子!色不異空,空不異色,色即是空,空即是色;

受、想、行、識,亦復如是。

「舍利子!是諸法空相,不生不滅,不垢不淨,不增不減。

是故,空中無色,無受、想、行、識;

無眼、耳、鼻、舌、身、意;無色、聲、香、味、觸、法;

無眼界,乃至無意識界;

無無明亦無無明盡,乃至無老死亦無老死盡;

無苦、集、滅、道;無智,亦無得。

「以無所得故,菩提薩,埵依般若波羅蜜多故,心無罣礙;

無罣礙故,無有恐怖,遠離顛倒夢想,究竟涅槃。

三世諸佛依般若波羅蜜多故,得阿耨多羅三藐三菩提。

「故知般若波羅蜜多是大神咒、是大明咒、是無上咒、是無等等咒,

能除一切苦,真實不虛,故說般若波羅蜜多咒」。

即說咒曰:「揭帝 揭帝 般羅揭帝 般羅僧揭帝 菩提僧莎訶」。
